\documentclass[10pt]{beamer}

% Theme settings
\usetheme{CambridgeUS}
\usecolortheme{default}
\usefonttheme{professionalfonts}
\setbeamertemplate{navigation symbols}{}
\setbeamertemplate{footline}[frame number]

% Packages
\usepackage{graphicx}
\usepackage{booktabs}
\usepackage{hyperref}
\usepackage{ragged2e}

% Metadata
\title[Firecracker vs Docker]{\textbf{Firecracker MicroVM vs Docker Container: A Benchmarking Study}}
\subtitle[OS Project]{Operating Systems Course Project}
\author[030, 047, 154]{202351030 \\ 202351047 \\ 202351154}
\institute[IIIT-V]{IIIT Vadodara \\ Gandhinagar Campus}
\date{November 2025}

\begin{document}

% ---------------------------
% Title Slide
% ---------------------------
\begin{frame}
  \titlepage
  \vfill
  \tiny Benchmarking cold start performance and resource usage between Docker containers and Firecracker microVMs.
\end{frame}

% ---------------------------
% Problem Statement
% ---------------------------
\begin{frame}{Problem Statement}
\justifying
\begin{itemize}
    \item \textbf{Problem:} Understanding performance trade-offs between traditional containerization (Docker) and lightweight virtualization (Firecracker)
    \item \textbf{Objectives:}
    \begin{itemize}
        \item Compare cold start times
        \item Measure CPU and memory resource usage
        \item Identify use-case specific advantages
    \end{itemize}
    \item \textbf{Technology Used:} 
    \begin{itemize}
        \item Docker Engine for containerization
        \item Firecracker v1.7.0 for microVMs
        \item Python for benchmarking automation
        \item KVM for hardware virtualization
    \end{itemize}
    \item \textbf{Motivation:} Modern cloud workloads require both fast startup times and strong isolation. Understanding when to use containers vs microVMs is crucial for optimizing serverless and multi-tenant environments.
\end{itemize}
\end{frame}

% ---------------------------
% System Architecture
% ---------------------------
\begin{frame}{System Architecture}
\begin{center}
\includegraphics[width=0.8\textwidth]{0.png}
\end{center}
\vspace{-3mm}
\footnotesize
Automated benchmarking workflow: User provides application → System builds/spawns both Docker container and Firecracker microVM → Runs cold start \& resource monitoring tests → Generates comparison report
\end{frame}

% ---------------------------
% Implementation Setup
% ---------------------------
\begin{frame}{Implementation Setup}
\justifying
\begin{itemize}
    \item \textbf{Test Environment:}
    \begin{itemize}
        \item OS: WSL2 (Ubuntu) with KVM support
        \item Docker Engine v24.0+
        \item Firecracker v1.7.0
        \item Python 3.9+ with docker, requests, psutil libraries
    \end{itemize}
    
    \item \textbf{Benchmark Components:}
    \begin{itemize}
        \item Cold start test: Simple Python HTTP server (port 8080)
        \item Resource monitoring: 10-second observation period
        \item Sample rate: 0.5 seconds (20 samples)
    \end{itemize}
    
    \item \textbf{Firecracker Configuration:}
    \begin{itemize}
        \item 1 vCPU, 512MB RAM
        \item Ubuntu 18.04 rootfs with custom init script
        \item TAP network interface (172.16.0.2/24)
    \end{itemize}
\end{itemize}
\end{frame}

% ---------------------------
% Cold Start Results
% ---------------------------
\begin{frame}{Results – Cold Start Performance}
\begin{center}
\includegraphics[width=0.75\textwidth]{coldstart_comparison.png}
\end{center}

\begin{table}[h]
\centering
\begin{tabular}{lcc}
\toprule
\textbf{Technology} & \textbf{Startup Time} & \textbf{Performance} \\
\midrule
Docker Container & 2.87 - 4.43 seconds & Baseline \\
Firecracker microVM & 1.84 - 0.64 seconds & \textbf{1.55-6.9x faster} \\
\bottomrule
\end{tabular}
\end{table}

\vspace{3mm}
\footnotesize
\textbf{Key Finding:} Firecracker achieves significantly faster cold starts due to direct kernel boot without container runtime overhead.
\end{frame}

% ---------------------------
% Resource Usage Results
% ---------------------------
\begin{frame}{Results – Resource Usage}
\begin{center}
\includegraphics[width=0.75\textwidth]{resource_comparison.png}
\end{center}

\begin{table}[h]
\centering
\footnotesize
\begin{tabular}{lccc}
\toprule
\textbf{Metric} & \textbf{Docker} & \textbf{Firecracker} & \textbf{Difference} \\
\midrule
Avg CPU Usage & 0-5\% & 0.98-3.00\% & Comparable \\
Avg Memory & 0-46 MB & 58-59 MB & +28\% (Firecracker) \\
Peak Memory & 0-52 MB & 58-59 MB & +15\% (Firecracker) \\
\bottomrule
\end{tabular}
\end{table}

\vspace{3mm}
\footnotesize
\textbf{Key Finding:} Docker containers have lower memory footprint due to shared kernel, while Firecracker includes guest OS overhead but provides VM-level isolation.
\end{frame}

% ---------------------------
% Trade-offs Analysis
% ---------------------------
\begin{frame}{Analysis – Technology Trade-offs}
\justifying

\begin{columns}[T]
\column{0.48\textwidth}
\textbf{Docker Containers:}
\begin{itemize}
    \item[+] Lower memory usage
    \item[+] Mature ecosystem
    \item[+] Better tooling support
    \item[−] Slower cold starts
    \item[−] Process-level isolation
\end{itemize}

\column{0.48\textwidth}
\textbf{Firecracker microVMs:}
\begin{itemize}
    \item[+] \textbf{1.5-6.9x faster startup}
    \item[+] VM-level isolation
    \item[+] Enhanced security
    \item[−] Higher memory footprint
    \item[−] More complex setup
\end{itemize}
\end{columns}

\vspace{5mm}
\textbf{Use Case Recommendations:}
\begin{itemize}
    \item \textbf{Serverless/FaaS:} Firecracker (fast cold starts critical)
    \item \textbf{Multi-tenant workloads:} Firecracker (stronger isolation)
    \item \textbf{Resource-constrained environments:} Docker (lower memory)
    \item \textbf{Traditional microservices:} Docker (ecosystem maturity)
\end{itemize}
\end{frame}

% ---------------------------
% Conclusion
% ---------------------------
\begin{frame}{Conclusion}
\justifying
\begin{itemize}
    \item \textbf{Key Contributions:}
    \begin{itemize}
        \item Automated benchmarking tool for fair comparison
        \item Quantified cold start performance: Firecracker 1.5-6.9x faster
        \item Measured resource overhead: Firecracker uses ~28\% more memory
        \item Provided use-case specific recommendations
    \end{itemize}
    
    \item \textbf{Limitations:}
    \begin{itemize}
        \item Tests conducted on single machine configuration
        \item Limited to HTTP server workloads
        \item Docker monitoring compatibility issues in WSL2
        \item Network overhead not extensively tested
    \end{itemize}
    
    \item \textbf{Future Work:}
    \begin{itemize}
        \item Benchmark diverse workloads (CPU-intensive, I/O-bound)
        \item Test on bare-metal systems and cloud platforms
        \item Add security isolation benchmarks
        \item Implement automated stress testing with varying loads
    \end{itemize}
\end{itemize}
\end{frame}

% ---------------------------
% References
% ---------------------------
\begin{frame}{References}
\footnotesize
\begin{thebibliography}{99}
\bibitem{firecracker} Firecracker Documentation, \url{https://firecracker-microvm.github.io/}
\bibitem{docker} Docker Documentation, \url{https://docs.docker.com/}
\bibitem{agache2020} Agache et al., "Firecracker: Lightweight Virtualization for Serverless Applications", NSDI 2020
\bibitem{psutil} psutil Library, \url{https://github.com/giampaolo/psutil}
\bibitem{repo} Project Repository: \url{https://github.com/thedevyashsaini/OS_Project_2025}
\end{thebibliography}
\end{frame}

% ---------------------------
% Thank You
% ---------------------------
\begin{frame}
\begin{center}
\Huge Thank You!

\vspace{10mm}

\Large Questions?

\vspace{15mm}

\normalsize
\textbf{GitHub:} \url{https://github.com/thedevyashsaini/OS_Project_2025}
\end{center}
\end{frame}

\end{document}
